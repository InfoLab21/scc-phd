\documentclass[12pt,a4paper]{article}
\usepackage[utf8]{inputenc}
\usepackage{hyperref}
\hypersetup{colorlinks=true, linkcolor=blue,  anchorcolor=blue,  
	citecolor=blue, filecolor=blue, menucolor=blue,  
	urlcolor=blue}
\usepackage{amsmath,amssymb,array,xcolor,multicol,verbatim}
\usepackage[normalem]{ulem}
\usepackage[pdftex]{graphicx}
\usepackage{soul}
\usepackage{fullpage}

\usepackage{tabularx,array}
%\newcolumntype{R}{>{\raggedleft\let\newline\\\arraybackslash\hspace{0pt}}X}

\setlength{\parskip}{6pt}
\setlength{\parindent}{0pt}

\title{PhD Appraisal Schedule (Part Time)} 
\author{\includegraphics[width=0.6\linewidth]{scc-logo.pdf}}
\date{Version 1.0 (10/02/2018)}

\makeatletter
\renewcommand{\@seccntformat}[1]{}
\makeatother

\begin{document}

\maketitle

\section{Introduction}
The School of Computing \& Communications is committed to providing effective PhD supervision and appraisal of PhD progress, leading ultimately to successful completion. Part-time PhD students are expected to submit their thesis after 4 years, with the University stipulating a maximum completion deadline of 7 years.

The School's PhD appraisal schedule is designed to assist in the monitoring of progress towards timely submission, and providing feedback and assessment above and beyond that provided through usual supervision.

This schedule is in line with the Faculty of Science and Technology PhD assessment, and adheres to
\href{https://gap.lancs.ac.uk/ASQ/QAE/MARP/Documents/PGR-Regs.pdf}{Lancaster University's Postgraduate Research Regulations} and 
\href{https://gap.lancs.ac.uk/ASQ/Policies/Documents/Postgraduate-Research-Code-of-Practice.pdf}{The Code of Practice for Postgraduate Research Programmes}.

The schedule of dates set out below are the default dates for a part-time PhD student. An individual's schedule may change, for example if there are periods of intercalation. You will receive reminders for each appraisal point via Moodle a month before the deadline.

\section{Progress Report}
The progress report, to be established in the first 4 months of the PhD, should be a single evolving document for the whole PhD process, and updated regularly. As a minimum, for every appraisal stage that follows, an updated copy of the report should be uploaded to Moodle, containing the following:
\begin{itemize}
	\item A brief summary (around half a page) of progress since the last appraisal stage and against the completion timeline, and any problems encountered.
	\item Updated assessment of training and development of skills needed, and a record of any training or development undertaken, including courses, presentations, teaching, etc.\
	\item An updated completion timeline with appropriate milestones and publication plans for the remainder of the PhD.
\end{itemize}

In addition, extra text or documents will be requested at different appraisal stages. A \LaTeX{} template is available for the progress report.

\section{Appraisal Panel}
By 4 months, every PhD student will be assigned an appraisal panel, consisting of:
\begin{description}
	\item[A Subject specialist(s)] A member of academic staff who is relatively close to the PhD research, but not involved in direct supervision. This could be a named supervisor, somebody who might be involved in collaborative work, or somebody from the same research group. In the event of an inter-disciplinary PhD (e.g.\ with supervisors from multiple departments) it is expected that there will be one panel member from each discipline.
	\item[A PGRC member] A member of academic staff from the SCC Postgraduate Research Committee, whose responsibility will be to ensure consistency and fairness across appraisals, and feedback issues and concerns to the School. This panel member shall normally be from a different research group than that of the PhD student and subject specialist.
\end{description}

Where possible, the appraisal panel will remain the same throughout the PhD, but replacements will be made if required (e.g.\ due to staff departures).

Please note, appraisal panel members are not permitted to serve as an examiner for the final PhD submission and viva.


\section{Feedback}
At each appraisal stage, the panel shall provide feedback on the PhD research, progress, and planning. The subject specialist shall lead feedback on the research element, with the PGRC member leading feedback on the progress and planning. Supervisors will approve each appraisal stage, but it is expected that their feedback will be given through the normal supervision process. Supervisors are expected to provide feedback on appraisal documents before submission, for which the PhD student and supervisors should agree a timeline in advance.

Feedback from each appraisal stage shall be provided within one month of document submission, or any panel meeting, whichever occurs latest. All feedback will be provided via Moodle.


\section{4 months}
The first review is designed to ensure that good foundations for the PhD have been established and appropriate planning has taken place. To this end, the following should have been completed by this stage:
\begin{enumerate}
\item Attended compulsory induction with the School, and ideally with the Faculty and the University.
\item Discussion with supervisor(s) about required training and development of skills, and an initial plan agreed for addressing these needs.
\item Initial research questions and direction developed.
\item A presentation to the department, presenting the research challenge and the plans for the PhD. This could be in the form of a poster presentation at the annual department PhD event, but could also be in another form (e.g. department seminar). For part-time students, this can be delayed until 10 months.
\item An agreed initial completion timeline, with appropriate milestones.
\item Research profile setup on \href{https://pure.lancs.ac.uk}{University research portal (Pure)}.
\end{enumerate}

\subsection{Submission}
The \textbf{progress report} should be started and submitted to Moodle by the first day of the 5th month of registration, confirming the above items have taken place, and providing details where necessary.


\section{10 months (annual review)}
An updated \textbf{progress report} to be submitted via Moodle by the first day of the 11th month of registration, alongside a second document containing an initial research proposal. The proposal should be 2-4 pages long and is intended as a starting point for the confirmation panel research proposal to be produced at 22 months. The precise structure of the proposal can differ between PhD students, but it should normally contain as a minimum:
\begin{itemize}
	\item A proposed title for the PhD.
	\item A section introducing the research topic, motivating the research, and setting out the research questions to be addressed.
	\item An initial literature review, covering key papers related to the research and highlighting novel aspects of the proposed research, along with a plan for a more substantial literature review to follow.
	\item A proposed methodology, describing how the research will be undertaken.
	\item Ethical considerations.
	\item A bibliography, listing references cited in the proposal (not included in page count).
\end{itemize}

A subsequent meeting will be arranged between the PhD student and appraisal panel members (expected within 1 month). Progress will be discussed along with plans for the next year of the PhD. There will also be an explicit discussion about supervision and feedback on department processes.

\section{16 months}
An updated \textbf{progress report} to be submitted via Moodle by the first day of the 17th month of registration, including a brief reflection on the annual review feedback.

\section{22 months (Confirmation Panel)}
The second annual review includes the Confirmation Panel for the PhD, this is an important formal assessment of the viability of the PhD, and initial progress. The assessment is based upon a \textbf{Research Proposal}, the updated \textbf{Progress Report}, and an appraisal panel meeting with a \textbf{Research Presentation}.

\subsection{Research Proposal}
An 8--10 page document which sets out the research to be carried out during the PhD. The precise structure of the proposal can differ between PhD students, but it should normally contain as a minimum:
\begin{itemize}
	\item A proposed title for the PhD.
	\item A section introducing the research topic, motivating the research, and setting out the research questions to be addressed.
	\item A substantial literature review, critically analysing related work, and clearly highlighting the novel aspects of the proposed research.
	\item A proposed methodology, describing how the research will be undertaken.
	\item Ethical considerations.
	\item A bibliography, listing references cited in the proposal (not included in page count).
\end{itemize}

\subsection{Submission}
The research proposal and updated progress report must be submitted via Moodle by the first day of the 23rd month of registration. The Confirmation Panel meeting will be scheduled as soon as possible after this (expected within one month).

\subsection{Confirmation Panel meeting}
A 10-minute \textbf{Research Presentation} should be prepared for the meeting, this will summarise the research proposal. The panel will ask questions arising from the presentation, research proposal and progress report. There will also be an explicit discussion about supervision and feedback on School processes. The PhD student will be asked to step out of the room whilst the panel discusses their assessment. The PhD student will be invited to return to discuss the outcome of the panel. The meeting should not last more than 1 hour in total.

\subsection{Outcomes}

\begin{description}
	\item[Confirm as PhD] If the appraisal panel is satisfied that the foundations exist for a viable PhD and satisfactory progress has been made, a recommendation will be made to confirm as PhD.
	\item[Defer decision] If after the first meeting the panel have reservations about the viability of the PhD, or major concerns about the quality of the work submitted, it will be recommended that a reassessment takes place within 4 months. The reassessment will be in the same form as the initial assessment, with both the Research Proposal and Progress Report resubmitted, addressing the feedback received in the first panel. The date of the resubmission and second panel will be agreed at the time of the first panel.
	
	There will only be one opportunity for reassessment. Following the second panel, the following options are available:
	\begin{description}
		\item[Confirm as PhD] If the appraisal panel is satisfied that the foundations exist for a viable PhD and the concerns raised have been addressed, a recommendation will be made to confirm as PhD.
		\item[Confirm as MPhil] If the concerns raised in the first panel have not been addressed to a satisfactory level, and the panel have significant concerns about the viability of the PhD, but believe the work could be completed as an MPhil, it can be recommended that the student transfer to an MPhil.
		\item[Recommend exclusion] If the concerns raised in the first panel have not been addressed to a satisfactory level, and the panel have significant concerns about the viability of the PhD, and believe the work could not be completed as an MPhil, it can be recommended that the student does not continue their studies.
	\end{description}
\end{description}

\subsection{Feedback}
After the first panel, and the second panel if applicable, the appraisal panel shall produce a report (1--2 pages) containing feedback and recommendations, and detailing any concerns raised. The subject specialist shall lead the writing of this report. This should be submitted via Moodle within a month of the panel meeting, and also to the Postgraduate Teaching Office along with the confirmation panel form (available from \url{http://www.lancaster.ac.uk/current-staff/registry/student-forms/}).


\section{30 months}
An updated \textbf{progress report} to be submitted via Moodle by the first day of the 31st month of registration, including a brief reflection on the confirmation panel feedback.


\section{36 months (annual review)}
An updated \textbf{progress report}, including a summary of the state of the research, to be submitted via Moodle by the first day of the 37th month of registration, alongside a second document containing an initial \textbf{thesis structure and plan}.

A subsequent meeting will be arranged between the PhD student and appraisal panel members (expected within 1 month). Progress will be discussed along with plans for the third year of the PhD. There will also be an explicit discussion about supervision and feedback on department processes.


\section{42 months}
An updated \textbf{progress report} to be submitted via Moodle by the first day of the 43rd month of registration, including a brief reflection on the annual review feedback.


\section{46 months (annual review)}
An updated \textbf{progress report} should be submitted to Moodle, this should include a summary of the state of the research and thesis. The completion timeline should be specific and detailed at this stage, with clear milestones and a proposed submission date. This should include planned and agreed periods for supervisor feedback on thesis chapters.

A subsequent meeting will be arranged between the PhD student and appraisal panel members (expected within 1 month), with a focus on the completion timeline. As necessary, a series of further appraisal stages will be scheduled and agreed to check on progress towards completion. This may include:
\begin{itemize}
	\item Progress reports
	\item Further panel meetings
	\item A research talk to the School
	\item A mock viva.
\end{itemize}


\section{Version History}
\begin{tabularx}{\textwidth}{llXX}
	\textbf{Version} & \textbf{Date} & \textbf{Detail} & \textbf{Author} \\ 
	1.0 & 10/02/2018 & Finalised first version. & Alistair Baron (PhD Tutor) \\ 
\end{tabularx} 


\end{document}